\begin{abstract}
Nowadays, web applications' needs are more sophisticated and demanding than in the past.
Efficiency and security are hence important points that must be addressed. However JavaScript,
the de facto standard of scripting languages on the web, is not able to meet these requirements.

In order to compensate for JavaScript's downsides, \textit{WebAssembly} (WASM) was introduced.
WASM is a new language that is ``designed for efficient execution and compact representation of code
on modern processors including in a web browser'' \cite{wasm-w3c-announcement}, and strives to
improve both performance and power consumption.
Although WASM was conceived to run in browsers, it can also be run directly on the host with the aid of runtimes, such as
\textit{Wasmtime} and \textit{Wasmer}, that do provide a fair level of security.
In some use cases, however, the provided security level is not enough and solutions to enhance it
are necessary.

The aim of this thesis work is to explore how these runtimes can be further restricted
through the aid of Linux Security Modules so to improve security and give more control to the user.
\end{abstract}