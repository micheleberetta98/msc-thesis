\section{Testing plan}

The main goals of this performance testing plan are:
\begin{itemize}
  \item to find out whether the usage of LSM implies some sort of performance penalty;
  \item to identify possible improvable points when embedding WebAssembly in another language.
\end{itemize}

\subsection{System and hardware}

The system and hardware used for both the development of the project described in Section \ref{sec:restricting-wasi-landlock}
and the performed test is as follows:
\begin{itemize}
  \item Arch Linux as the operating system, more specifically the 2022.05.01 version;
  \item an Intel Core i5-7200U quad-core with a clock rate of 2.5 GHz;
  \item 8 GB of RAM;
  \item a 120 GB solid state disk.
\end{itemize}

The choice of the operating system is mainly dictated by the fact that Arch Linux has both Landlock and eBPF active
out of the box, removing the need to compile the Linux kernel with the necessary flags to enable these
functionalities.

\subsection{Performance indicators}

The main performance indicators will be the mean execution time, measured in milliseconds, together with its
standard deviation in order to compensate for variability.
These measures are always obtained from a sample of 100 runs, executed and measured by \texttt{hyperfine},
a command-line benchmarking tools \cite{hyperfine}.

\section{Comparison between different methods}

\subsection{Landlock and eBPF on a native binary}

\subsection{Landlock and eBPF on a WebAssembly binary}

\section{Comparison between a restricted and an unrestricted binary}

\subsection{Landlock on a WebAssembly binary}

\subsection{eBPF on a WebAssembly binary}

\section{Internal analysis of the developed project}

\subsection{Performance impact of Landlock}

\subsection{Performance impact of using a WASI library}