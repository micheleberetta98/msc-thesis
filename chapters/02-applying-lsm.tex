\section{Goals}

The main goal of using \textit{Linux Security Modules} in combination with WebAssembly
is to provide a finer grain customisation when choosing how to limit the permissions given
to the executable.

More specifically, the desired features are
\begin{enumerate}
  \item to give the possibility to differentiate between directories and files when giving permissions;
  \item to clearly separate read, write, execution and deletion capabilities;
  \item eventually, to prevent the creation of new files, directories and other objects (such as symbolic links).
\end{enumerate}

Additional desirable features include the possibility to apply exceptions to a certain rule, as well
as limiting the access to other resources in the operating systems, such as network, devices, and
inter-process communications.

Finally, usability is also taken into consideration, especially from the point of view of a user.

\section{Landlock}

\subsection{Code architecture}

In order to use Landlock, it must be enabled when compiling Linux.

The code itself is written in \textit{Rust}, and makes use of the
\textit{rust-landlock}\footnote{\url{https://github.com/landlock-lsm/rust-landlock}} crate
to communicate with Landlock and the \textit{wasmtime} and \textit{wasmtime-wasi} official
crates in order to have a runtime environment to execute a WebAssembly binary.

The project is divided into five modules, which are:
\begin{enumerate}
  \item \texttt{args}, used to define and parse the command line arguments that specify the WebAssembly binary,
        the directories to preopen, and the allowed privileges on individual folders and/or files;
  \item \texttt{landlock}, which handles the creation and update of the permissions' ruleset;
  \item \texttt{main}, the module that interfaces with the user;
  \item \texttt{path\_access}, containing a helper structure to track permission for a single path;
  \item and finally \texttt{wasm}, that communicates with the \textit{wasmtime} and the \textit{wasmtime-wasi} crates
        in order to run the provided binaries, as well as preopening directories.
\end{enumerate}

\subsubsection{The \texttt{args} module}

The main definition of this module is the \texttt{Args} struct.

\begin{code}[language=rust]
  pub struct Args {
    pub wasm_module: String,

    pub dirs: Vec<String>,

    pub mapdirs: Vec<(String, String)>,

    pub fs_allows: Vec<(String, BitFlags<AccessFs>)>,

    pub no_landlock: bool,
  }
\end{code}

Every single field is mapped to one command line argument, which can appear multiple (or no) times
if it is a \texttt{Vec}. Their usage and meaning is described in Table \ref{table:landlock-cli-args}.

Note that here preopening a directories does not mean it and its contents will be accessible by the
provided binary at runtime - they can still be restricted by the landlock policies. Preopening is
necessary though in order to give the possibility to access its contents.

Directories can also me \textit{mapped} - in a way similar to the one provided by \textit{wasmtime} command line tool,
it is possible to remap path, so that it's possible to have the binary access any arbitrary path in the system
without manually copying and/or executing it in a specific directory.

\begin{table}[h]
  \centering
  \begin{tabular}{|l|l|l|l|}
    \hline
    \textit{Argument} & \textit{Required} & \textit{Description} \\
    \hline\hline
    \texttt{wasm\_module} & Yes & The path to the WASM binary to run \\ \hline
    \texttt{dir} & No & All the directories to preopen \\ \hline
    \texttt{mapdir} & No & Eventual mappings between the directories \\ \hline
    \texttt{fs\_allow} & No & All the permitted actions on a single path \\ \hline
    \texttt{no\_landlock} & No & Used to disable the self restriction done by landlock \\
    \hline
  \end{tabular}
  \caption{All the available command line arguments}
  \label{table:landlock-cli-args}
\end{table}

The arguments \texttt{dir}, \texttt{mapdir} and \texttt{fs\_allow} can appear multiple time, and all the values
are then collected into a single \texttt{Vec}. In this way, it is possible to apply different restrictions on different
paths - for example, one could set some files as read-only, while another subset of files in a specific directory
could also be writable.

The \texttt{fs\_allow} is made up of two parts - a path to restrict, and a series of enabled permissions on that specific path.
Each permission is mapped one-to-one to the flags provided by Landlock in a manner listed in Table \ref*{table:landlock-flags}.
The specific ABI used is the first version, so at the moment it is not possible to specify a \texttt{LANDLOCK\_ACCESS\_FS\_REFERER}
in order to reparent a file hierarchy.

Lastly, the \texttt{no\_landlock} argument is only for testing purposes - it is off by default, so that
landlock is enabled and self restriction is applied if not explicitly disabled.

\begin{table}[h]
  \centering
  \begin{tabular}{|l|l|}
    \hline
    \textit{Flag} & \textit{Enabled permission} \\ \hline\hline
    \texttt{X} & Execute a file \\ \hline
    \texttt{W} & Write to a file \\ \hline
    \texttt{R} & Read a file \\ \hline
    \texttt{RDir} & Open a directory or list its content \\ \hline
    \texttt{DDir} & Delete an empty directory or rename one \\ \hline
    \texttt{D} & Unlink or rename a file \\ \hline
    \texttt{MChar} & Create, rename or link a character device \\ \hline
    \texttt{MDir} & Create or rename a directory \\ \hline
    \texttt{MReg} & Create, rename or link a regular file \\ \hline
    \texttt{MSock} & Create, rename or link a socket \\ \hline
    \texttt{MFifo} & Create, rename or link a named pipe \\ \hline
    \texttt{MBlock} & Create, rename or link a block device \\ \hline
    \texttt{MSym} & Create, rename or link a symbolic link \\ \hline
  \end{tabular}
  \caption{Provided landlock flags}
  \label{table:landlock-flags}
\end{table}

\subsection{Available permission settings}

\subsection{Advantages and disadvantages}

\section{eBPF}

\subsection{Architecture}

\subsection{Available permission settings}

\subsection{Advantages and disadvantages}