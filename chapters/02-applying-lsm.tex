\section{Goals}

The main goal of using \textit{Linux Security Modules} in combination with WebAssembly
is to provide a finer grain customisation when choosing how to limit the permissions given
to the executable.

More specifically, the desired features are
\begin{enumerate}
  \item to give the possibility to differentiate between directories and files when giving permissions;
  \item to clearly separate read, write, execution and deletion capabilities;
  \item eventually, to prevent the creation of new files, directories and other objects (such as symbolic links).
\end{enumerate}

Additional desirable features include the possibility to apply exceptions to a certain rule, as well
as limiting the access to other resources in the operating systems, such as network, devices, and
inter-process communications.

Finally, usability is also taken into consideration, especially from the point of view of a user.

\newpage
\section{Landlock}

\subsection{Code architecture and description}\label{sec:landlock-code-architecture}

The code itself is written in \textit{Rust}, and makes use of the
\textit{rust-landlock}\footnote{\url{https://github.com/landlock-lsm/rust-landlock}} crate
to communicate with Landlock and the \textit{wasmtime} and \textit{wasmtime-wasi} official
crates in order to have a runtime environment to execute a WebAssembly binary.

The project is divided into five modules, which are:
\begin{enumerate}
  \item \texttt{args}, used to define and parse the command line arguments that specify the WebAssembly binary,
        the directories to preopen, and the allowed privileges on individual folders and/or files;
  \item \texttt{landlock}, which handles the creation and update of the permissions' ruleset;
  \item \texttt{main}, the module that interfaces with the user;
  \item \texttt{path\_access}, containing a helper structure to track permission for a single path;
  \item and finally \texttt{wasm}, that communicates with the \textit{wasmtime} and the \textit{wasmtime-wasi} crates
        in order to run the provided binaries, as well as preopening directories.
\end{enumerate}

\begin{figure}[h]
  \centering
  \includegraphics[width=0.9\linewidth]{rust-landlock-code-diagram.png}
  \caption{The code's architecture}
  \label{fig:rust-landlock-code-architecture}
\end{figure}

The architecture outline is visible in Figure \ref{fig:rust-landlock-code-architecture}, where
are outlined the most important external crates and all dependencies between the modules, represented
by directed arrows.

\subsubsection{The \texttt{args} module}

The main definition of this module is the \texttt{Args} struct, visible in Listing \ref{lst:arg-struct}.

Every single field is mapped to one command line argument thanks to the \textit{clap} crates which handles the
creation and parsing of the CLI arguments structure.
Their usage and meaning is described in Table \ref{table:landlock-cli-args}.

Note that here preopening a directories does not mean it and its contents will be accessible by the
provided binary at runtime - they can still be restricted by the landlock policies. Preopening is
necessary though in order to give the possibility to access its contents.

Directories can also be \textit{mapped} - in a way similar to the one provided by the \textit{Wasmtime} command line tool,
it is possible to remap path, so that it's possible to have the binary access any arbitrary path in the system
without manually copying and/or executing it in a specific directory.

\begin{code}[language=rust, caption=The \texttt{Args} struct, label=lst:arg-struct]
  #[derive(Parser, Debug)]
  #[clap(author, version, about, long_about = None)]
  pub struct Args {
    // The module to execute
    pub wasm_module: String,

    // The preopepend dir(s) to pass to wasmtime
    pub dirs: Vec<String>,

    // The preopened mapped dir(s) to pass to wasmtime
    pub mapdirs: Vec<(String, String)>,

    // A list of the allowed privileges on a particular folder/file
    pub fs_allows: Vec<(String, BitFlags<AccessFs>)>,

    pub no_landlock: bool,
  }
\end{code}

\begin{table}
  \centering
  \begin{tabular}{|l|l|l|l|}
    \hline
    \textit{Argument} & \textit{Required} & \textit{Description} \\
    \hline\hline
    \textit{WASM module} & Yes & The path to the WASM binary to run \\ \hline
    \texttt{dir} & No & All the directories to preopen \\ \hline
    \texttt{mapdir} & No & Eventual mappings between the directories \\ \hline
    \texttt{fs-allow} & No & All the permitted actions on a single path \\ \hline
    \texttt{no-landlock} & No & Used to disable the self restriction done by landlock \\
    \hline
  \end{tabular}
  \caption{All the available command line arguments}
  \label{table:landlock-cli-args}
\end{table}

The first argument is positional, and is the path where the WebAssembly module is located.

The arguments \texttt{dir}, \texttt{mapdir} and \texttt{fs-allow} can appear multiple times (or no times at all), and all the values
are then collected into a single \texttt{Vec}. In this way, it is possible to apply different restrictions on different
paths - for example, one could set some files as read-only, while another subset of files in a specific directory
could also be writable.

The \texttt{fs-allow} is made up of two parts, separated by a colon - a path to apply the restrictions to,
and a series of enabled permissions on that specific path, separated by a comma.
Each permission is mapped one-to-one to the flags provided by Landlock in a manner listed in Table \ref{table:landlock-flags}.
Moreover, there are three useful ``shortcuts'' used to represent common situations - \texttt{$\sim$read}, \texttt{$\sim$write} and \texttt{*}.

Lastly, the \texttt{no-landlock} argument is only for testing purposes - it is off by default, so that
landlock is enabled and self restriction is applied if not explicitly disabled.

\subsubsection{The \texttt{wasm} module}

This module manages a single WebAssembly binary module and forms a bridge to the \textit{wasmtime} and the \textit{wasmtime-wasi}
crates. In this project, \textit{Wasmtime} bindings were used instead of the \textit{Wasmer} ones, but both provide the
same level of functionality, albeit with different names and structure.

Here a WASM binary is represented by a vector of bytes, and it is also handled the creation
and construction of a \textit{WASI Context}, a structure defined by the \textit{wasmtime-wasi} crate used to store
all preopened directories, eventual imports and more that will be useful to run the WebAssembly module.

Here are defined various helpers to preopen directories, both unmapped and mapped.
Most importantly, in this module it's defined how to run a WebAssembly binary - it must defined both an \textit{engine} and a \textit{linker},
as well as a \textit{store} which represents the memory of a WebAssembly module described in the previous chapter.

Note that the WebAssembly module has to have a \textit{default exported function} to run - when compiling with a WASI target,
this is usually represented by the \texttt{\_start} function, which corresponds to the \texttt{main} function in languages
such as C and Rust.

\begin{code}[language=Rust, caption=The outline of the \texttt{wasm} module]
pub struct WasmModule {
  bytes: Vec<u8>, ctx_builder: WasiCtxBuilder,
}

impl WasmModule {
  // Reads the WASM module and initializes the WasiCtxBuilder
  pub fn new(path: &str) -> Result<Self> { ... }

  // Make inherit stdio to the WASM module
  pub fn use_stdio(mut self) -> Self { ... }

  // Preopen all given directories
  pub fn preopen_all(
    mut self,
    dirs: &Vec<String>) -> Result<Self> {...}

  // Preopen and map all given directories
  pub fn preopen_all_map(
    mut self,
    mapdirs: &Vec<(String, String)>) -> Result<Self> {...}

  // Preopen (and map) a single directory
  pub fn preopen(
    mut self,
    dir: &str, guest_path: &str) -> Result<Self> {...}

  // Run the module
  pub fn run(self) -> Result<()> {...}
}  
\end{code}

\subsubsection{The \texttt{landlock} and \texttt{path\_access} modules}

The \texttt{landlock} module is a thin wrapper around the API made available by the \textit{rust-landlock} crate.
It handles mostly ruleset creation, rule insertion, and enforces the policies before the desired WASM module is run.
The main outline of the code is visible in Listing \ref{lst:rust-landlock}.

It makes use of another thin wrapper, the \texttt{PathAccess} struct defined in the \texttt{path\_access} module.
In this module, the main entity is the \texttt{PathAccess} struct, that tracks which flags as defined in Table \ref{table:landlock-flags}
are applied to a single path. The flags list starts empty, and multiple flags can be added at any times.
Moreover, this module also takes care of conversions between types required by the Landlock ABI in order
to make the code compile.

\begin{code}[language=Rust, caption=The outline of the \texttt{landlock} module, label=lst:rust-landlock]
pub struct Landlock {
  ruleset: RulesetCreated,
}

impl Landlock {
  // Creates a new ruleset with flags from Landlock ABI version 1
  pub fn new() -> Result<Self, RulesetError> { ... }

  // Add a set of rules to the ruleset
  pub fn add_rules(
    mut self,
    rules: impl Iterator<Item = PathAccess>) -> Result<Self>
  { ... }

  // Add a single rule to the ruleset
  pub fn add_rule(
    mut self,
    path_access: PathAccess) -> Result<Self>
  { ... }

  // Self restrict the process and checks wether Landlock
  // is supported or not by the running kernel
  pub fn enforce(self) -> Result<RestrictionStatus> { ... }
}
\end{code}

\subsection{Available permission settings}

The program allows to specify all access right flags from the first versione of the Landlock
ABI\footnote{At the moment it is not possible to specify the \texttt{LANDLOCK\_ACCESS\_FS\_REFERER} in order to reparent a file hierarchy.}.
These flags are used in the command line arguments and the following examples illustrate how to combine them:
\begin{itemize}
  \item \texttt{-{}-dir ``.'' -{}-fs-allow ``input:R''} runs \texttt{bin.wasm} with only a read permission on the file \texttt{input};
  \item \texttt{-{}-dir ``.'' -{}-fs-allow ``input:X''} only allows the execution of \texttt{input};
  \item \texttt{-{}-dir ``.'' -{}-fs-allow ``.:MReg''} only allows the creation, rename or linking of regular files in the current directory;
  \item \texttt{-{}-dir ``.'' -{}-dir ``./inner'' -{}-fs-allow ``.:R'' -{}-fs-allow ``./inner:*''} combines multiple access rights so that
        files in the current directory can only be read, while there are no restrictions only in a specific subdirectory (in this case, \texttt{inner}).
\end{itemize}

\begin{table}
  \centering
  \begin{tabular}{|l|l|}
    \hline
    \textit{Flag} & \textit{Enabled permission} \\ \hline\hline
    \texttt{X} & Execute a file \\ \hline
    \texttt{W} & Write to a file \\ \hline
    \texttt{R} & Read a file \\ \hline
    \texttt{RDir} & Open a directory or list its content \\ \hline
    \texttt{DDir} & Delete an empty directory or rename one \\ \hline
    \texttt{D} & Unlink or rename a file \\ \hline
    \texttt{MChar} & Create, rename or link a character device \\ \hline
    \texttt{MDir} & Create or rename a directory \\ \hline
    \texttt{MReg} & Create, rename or link a regular file \\ \hline
    \texttt{MSock} & Create, rename or link a socket \\ \hline
    \texttt{MFifo} & Create, rename or link a named pipe \\ \hline
    \texttt{MBlock} & Create, rename or link a block device \\ \hline
    \texttt{MSym} & Create, rename or link a symbolic link \\ \hline
    \texttt{$\sim$read} & Combination of \texttt{X}, \texttt{R} and \texttt{RDir} \\ \hline
    \texttt{$\sim$write} & Combination of all but \texttt{X}, \texttt{R} and \texttt{RDir} \\ \hline
    \texttt{*} & Enable all flags \\ \hline
  \end{tabular}
  \caption{Provided Landlock flags}
  \label{table:landlock-flags}
\end{table}

The Landlock self-restriction is applied after the necessary preopening of the directories,
otherwise the running process would have to always have the permissions for listing directories even if
not needed by the WASM binary.

\subsection{Advantages and disadvantages}

\section{eBPF}

\subsection{Architecture}

\subsection{Available permission settings}

\subsection{Advantages and disadvantages}