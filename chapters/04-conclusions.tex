\section{WebAssembly and LSM}

\textit{WebAssembly}, abbreviated to WASM, is a relatively new binary instruction format that
strives to bring security, speed, portability, and compactness to the Web
as a low-level alternative to JavaScript.
With the advent of the \textit{WebAssembly System Interface} (WASI),
this language can also be treated as a compilation target for languages such as C and Rust,
and can be executed directly on various operating systems
through the aid of runtimes, such as \textit{Wasmtime} and \textit{Wasmer}.

However, these runtimes' CLI interfaces do not provide extensive user control
when dealing with the program's access rights to specific file system locations.
These interfaces can be extended by using pre-existing access control frameworks,
specifically the \textit{Linux Security Modules} framework.
Two frameworks in particular are taken into consideration -- \textit{Landlock},
to provide unprivileged scoped access control, and \textit{eBPF}, a virtual
machine that can run sandboxed program in a privileged context.

\section{The developed project}

This thesis work deals with the development of a proof-of-concept WASM runtime that
provides the means to extend a WASM binary's access rights specifiable by the user.
The development of the project was divided into three main phases.

Firstly, an analysis of the existing runtimes, \textit{Wasmtime} and \textit{Wasmer},
was required to understand how they behave in different situations and what
specific means they give to the user in order to restrict a WASM binary.
As highlighted in Section \ref{sec:introduction-wasi}, when using the default command line
interface, there is a lack of a fine granularity on what a program can do -- for example,
the user cannot describe what a WASM binary is permitted to do on a preopened directory's files.

Secondly, the two LSM frameworks in question, \textit{Landlock} and \textit{eBPF}, were studied and understood
so that they could be integrated with arbitrary code.
For this phase, having a mainstream Linux distribution is not enough -- Landlock and eBPF have to be enabled
to be used by programs. This can be done either by manually compiling the kernel with the desired
flags enabled, or by using particular distributions that have them enabled by default, such as Arch Linux.

Lastly, WASI libraries and the chosen LSM frameworks had to be integrated in a new runtime, described in
Chapter \ref{chap:restricting-wasi}, able to run WASM binaries while simultaneously sandboxing them according to the user's choice.
When dealing with Landlock, the integration with a given program is provided by C libraries,
which can also be wrapped in the language of one's choice -- for this project, this language is Rust,
in order to have a safer language that could still produce reasonably fast binaries.
On the other hand, eBPF has to be integrated externally, as the sandboxing action is done by
a server process, acting as a virtual machine.
Between the two, Landlock is definitely the easier one, both from a user's point of view,
regarding the available access control options, and from a developer's point of view, since
the existing libraries are quite straightforward to use. Moreover, its usage does not need the running program
to have root privileges.
The eBPF sandbox, although being more complex and requiring root privileges, is however
the more powerful one, allowing a great deal of fine tuning on multiple areas,
such as file system, network and inter-process communication.

Regarding performance, WASM is still not close to native speeds, and the usage of WASI libraries
brings a visible overhead against the more traditional runtimes, as shown in the tests in Chapter \ref{chap:performance}.
However, LSMs themselves are not a hindrance on performance if considered in isolation.

\section{Possible improvements and future work}

The first problem that could be addressed is the performance of the WASI libraries.
In the testing runs, a very high percentage of the time is spent into loading and running the WASM module --
acting as a bottleneck, these libraries are thus the first and most effective point to improve.
This could be done either through optimisations in the original code, or through a more thorough
configuration of the library's execution capabilities.

Another possible improvement for the existing WASM runtimes could be the integration of \textit{Landlock}
if the development of a custom full-fledged access control framework turns out to be too hard to implement.
This could effectively bring a greater deal of control directly available to the user when
accessing the file system, while still being relatively simple to use.
However, Landlock is only available on Linux, and moreover it has to be enabled in the kernel.
This would pose some problems -- the runtime code would be less portable,
since a custom version for Linux would have to be created, requiring
the creation of more OS-specific code and making development more difficult.
Furthermore, not all Linux users could take advantage of it, since enabling it can be a challenge that only more advanced users
could be able and/or willing to face\footnote{This could be eased if Landlock were to be already enabled in most mainstream distributions.}.
